\section{Networking}
\begin{frame}{Networking in Rust}
    \pause
    Networking can be done using the \mintinline{rust}|std::net| standard
    library
    \pause
    \begin{itemize}[<+->]
        \item \mintinline{rust}|TcpListener| --- TCP Socket Server
        \item \mintinline{rust}|TcpStream| --- Stream between local and remote socket
        \item \mintinline{rust}|UdpSocket| --- Functionality for UDP communication
        \item \mintinline{rust}|IpAddr| --- IPv4 and IPV6 addresses
        \begin{itemize}[<+->]
            \item \mintinline{rust}|Ipv4Addr|
            \item \mintinline{rust}|Ipv6Addr|
        \end{itemize}
        \item \mintinline{rust}|SocketAddr| --- socket address of IPv4 and IPv6
        \begin{itemize}[<+->]
            \item \mintinline{rust}|SocketAddrV4|
            \item \mintinline{rust}|SocketAddrV6|
        \end{itemize}
    \end{itemize}
\end{frame}

\begin{frame}{\mintinline{rust}|TcpListener|}
    \inputrust{code/tcplistener.rs}    
\end{frame}

\begin{frame}{\mintinline{rust}|TcpStream|}
    \inputrust{code/tcpstream.rs}    
\end{frame}

\begin{frame}{\mintinline{rust}|UdpSocket|}
    \inputrust{code/udpsocket.rs}
\end{frame}

\begin{frame}{Comparison with C}
    \resizebox{0.49\textwidth}{!}{\inputrust{code/server.rs}}
    \resizebox{0.49\textwidth}{!}{\inputc{code/server.c}}
\end{frame}